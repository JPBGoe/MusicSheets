\documentclass[10pt]{scrartcl}

% Papierformat und Seiten
\usepackage[paper=a4paper]{geometry}

% Spracheinstellungen
\usepackage[utf8]{inputenc}
\usepackage[ngerman]{babel}
\usepackage[T1]{fontenc}

% Glossar und Literatur


% Dokumenteigenschaften
\title{Pflichtenheft V00-01}
\author{Jan-Philipp Burchert}
\date{ }



\begin{document}
\maketitle

 
\tableofcontents
\newpage
\section{Einleitung}
Dieses Pflichtenheft richtet sich an die Entwickler, die diese Software warten und weiterentwickeln. Außerdem soll dieses Dokument die Funktionen beschreiben, die diese Software beinhaltet. Somit ist dieses Dokument auch für die Anwerder geeignet.

Dieses Dokument ging aus dem Lastenheft zu diesem Projekt hervor. Es beschreibt die Ziele eines Entwicklungskprojekts.

\section{Zielbestimmung}
Diese Software soll Informationen über Musikstücke und die Noten, die zu den Musikstücken gehören, verwalten. Neben dem Hinzufügen  und Ablegen von Noten, müssen diese auch für einzelne Musiker oder Register zusammengestellt werden. Neben dem Kopieren in einen Ausgabeordner wäre es wünschenswert, die Noten auch direkt über einen Drucker ausdrucken zu können. Jedoch soll diese Software nicht die Notenblätter selbst erstellen. Vielmehr soll die Software einen Dateibaum mit den Notenblättern verwalten  bzw. die Verwaltung erleichtern.

\section{Produkteinsatz}
\subsection{Anwendungsbereich}
Verwaltung
\subsection{Zeilgruppen}
Die Zielgruppe für diese Software sind diejenigen Musiker und Vereinsmitglieder, welche für die Pflege und Verwaltung der Musikstücke des Vereins verantwortlich sind. Dies sind v.a. die Mitglieder des Vorstands.
\subsection{Betriebsbedingungen}
Diese Software soll auf einem Laptop eingesetzt werden und wird immer dann benutzt, wenn Veränderungen am Notenbestand stattfinden oder wenn Noten an Musiker ausgegeben werden sollen.

\section{Produktübersicht}
Diese Software verwendet die Model-View-Controller-Architektur und wird als ausführbares Java jar-File zur Verfügung gestellt. Dieses jar-File beinhaltet für jedes dieser drei Bereiche ein eigenens Paket mit dem entsprechenden Programmcode. 

Im Packet \glqq Model\grqq~ liegen verschiedene Klassen, deren Attribute Zusatzinformationen zu den im Dateibaum hinterlegten Notenblättern und Musikstücke sind. Desweiteren sind hier relative Pfade zu den pdf der Notenblätter hinterlegt. Instanzen dieser Klassen stellen die Verwaltungsinformationen der Software dar.

Im Paket \glqq View\grqq~werden die Benutzeroberflächen der Software hinterlegt. Neben dem Hauptmenü mit seinen Optionen, sind dort auch Klassen vorhanden, die die benötigten Frames und Panels aus des Hauptmenüs realisieren.

Im Paket \glqq Controller\grqq~werden die Klassen hinterlegt, deren Instanzen für die Umsetzung der Benutzereingaben auf die Model-Ebene verantwortlich sind. Diese Instanzen werden werden von den View-Instanzen gerufen. 

\section{Produktfunktionen}
\begin{enumerate}
\item[F10] Hinzufügen, Bearbeiten und Löschen von Musikstücken
\item[F20] Hinzufügen, Bearbeiten und Löschen von Notenblättern
\item[F30] Erstellen, Speichern, Laden und Löschen eines Auftrittsprogramms
\item[F40] Kopieren einer Menge von ausgewählten Musikstücken aus dem Datenbestand in einen Ausgabeordner
\item[LF50] Suche von Musikstücken und Noten. Dabei soll sowohl eine Generelle Suche als auch eine Suche, bei der die einzelnen Kathegorien mit einbezogen werden möglich sein.
\end{enumerate}
\section{Produktübersicht}
\section{Produktdaten}
\begin{enumerate}
	\item[D10] Zu jedem Musikstück sind folgende Daten zu speichern: Name, Komponist, Arrangeur, Jahr, Verlag
	\item[D20] Zu jedem Musikstück kann der Benutzer weitere Informationen als Text ablegen.
	\item[D30] Zu jedem Notenblatt wird der Speicherort, das zu verwendene Instrument sowie dessen Stimmung als auch die Stimme innerhalb des Musikstücks gespeichert.
	\item[D40] Jedes Notenblatt gehört zu genau einem Musikstück.
\end{enumerate}
\section{Produktleistungen}
\begin{enumerate}
\item[L10] Kein Musikstück ist doppel im Sytem.
\item[L20] Kein Notenblatt ist innerhalb eines Musikstücks doppelt.
\item[L30] Eine Suche sollte nicht länger als 5 Sekunden dauern.
\item[L40] Sollte ein Musikstück oder ein Notenblatt aus dem System entfernt werden, so soll dies vollständig geschehen.
\item[L50] Das GUI sollte intuitiv verständlich sein.
\item[L60] Sowohl unter der Rubrik \glqq Musikstücke\grqq~  als auch \glqq Noten\grqq~ sollte die Suchfunktion ausführbar sein.
\item[L70] Die Software sollte möglichst unabhängig vom verwendeten Betriebssystem sein.
\item[L80] Die Notenblätter und Musikstücke sollten menschenlesbar einem Dateisystem liegen.
\end{enumerate}
\section{Qualitätsanforderungen}
Die Qualitätsanforderungen an dieses Projekt sind in Tabelle \ref{Tab. Qualitaetsanforderungen} aufgeführt.
\begin{table}
\centering
\begin{tabular}{| c | c | c | c | c |}
\hline
Produktqualität &	sehr gut&	gut & normal& irrelevant \\
\hline
\hline
Funktionalität	&			&	x	&		&		\\
\hline
Zuverlässigkeit	&			&		&	x	&		\\
\hline
Benutzbarkeit	&	x		&		&		&		\\
\hline
Effizienz		&			&		&	x	&		\\
\hline
Änderbarkeit	&			&		&		&	x	\\
\hline
Portierbarkeit	&			&	x	&		&		\\
\hline
\end{tabular}
\caption{Qualitätsanforderungen an das aktuelle Projekt.}
\label{Tab. Qualitaetsanforderungen}
\end{table}
\section{Benutzungsoberfläche}
Die Benutzeroberfläche besteht aus einem Hauptfenster, welches Reiter für \glqq Musikstücke\grqq, \glqq Noten\grqq~besteht. Hierbei sind innerhalb der Reiter die Optionen als betätigbare Knöpfe hinterlegt. Betätigung eines der Knöpfe öffnet das Panel zu dieser Option.
\section{Nichtfunktionale Anforderungen}
- keine -
\section{Technische Produktumgebungen}
\subsection{Hardware}
Bei der Hardware, auf der diese Software laufen soll, handelt es sich um einen aktuellen Laptop, welcher nicht sehr leistungsstark sein muss.
\subsection{Software}
Java muss installiert sein.
\subsection{Orgware}
-keine-

\section{Entwicklungsumgebung}
\subsection{Hardware}
Diese Software wird auf einem Laptop entwickelt.
\subsection{Software}
Die Entwicklungsumgebung ist Linux mit dem Editor vim.  
\subsection{Orgware}
Als Compiler wird Javac verwendet. Als Buildtool wird Apache Ant verwendet und als Analysewerkzeug checkstyle.

\section{Gliederung in Teilprodukte}
Es findet keine Gliederung statt.
\section{Glossar}
\paragraph*{Ausgabe} Diese Software gibt Notenblätter, die durch den Nutzer herausgesucht wurden in einen entsprechend benannten Ausgabeorder aus. Ausgabe wird in desem Dokument stellvertretend für diesen Vorgang verwendet.
\paragraph*{digitalisierte Speicherung} Ablage von (Noten-) Blätter auf einem Computer. Dazu werden die Blätter in Dateien (z.B. PDF) überführt und innerhalb eines Dateisystems in einem Baum abgelegt.
\paragraph*{Eingabe} Eingabe bezeichnet in diesem Kontext das Hinzufügen von Musikstücken oder Notenblättern in dieses System.
\paragraph*{GUI} Graphische Benutzeroberfläche (Graphical User Interface) zur Interaktion des Systems mit dem Benutzer.
\paragraph*{Musikstück} Musikalisches Werk, welches aus Notenblättern für verschiedene Instrumente besteht  und einem Künstler/ Komponist zuzuordnen ist.
\paragraph*{Noten} Noten werden in diesem Zusammenhang gleichbedeuten zu Notenblatt verwendet.
\paragraph*{Notenblatt} Notenblatt ist die Verwaltungsinstanz, die die Noten sowie alle zugehörigen Informationen zu den Notenblättern speichert bzw. verwaltet.
\paragraph*{Nutzer} Die Person, die diese Software auf einem Rechner verwendet. Hier häufig der Notenwardt oder Dirigent.
\paragraph*{Software} Gesamtheit aller Dokumente, Quellcodes and Bytecode, welche für die ordnungsgemäße Funktionsweise dieser Anwendung nötig sind.
\paragraph*{Stimmung} Verschiedene Instrumente gleichen Types können unterschiedliche Stimmungen haben. Eine Stimmung gibt an, wie ein gegriffenes C auf dem Instrument im Vergleich zu einem gegriffenen C auf einem Klavier klingt.
\paragraph*{Stimme} Die zu spielenden Harmonien und Rythmen eines Musikstücks, welche von einem Instrument gespielt werden sollen, können unter mehreren dieser Instrumente aufgeteilt werden. Jede Aufteilung ist eine Stimme für das Instrument innerhalb des Musikstücks. 



\end{document}
